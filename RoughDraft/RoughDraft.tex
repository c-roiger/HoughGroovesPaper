\documentclass[12pt]{article}
\usepackage{amsmath,amssymb,mathrsfs,fancyhdr,syntonly,lastpage,hyperref,enumitem,graphicx}

\usepackage[thmmarks,thref]{ntheorem}

\theoremstyle{nonumberplain}
\theoremheaderfont{\itshape}
\theorembodyfont{\upshape}
\theoremseparator{.}
\theoremsymbol{\ensuremath{\square}}
\newtheorem{solution}{Solution}

\hypersetup{colorlinks=true,urlcolor=black}

\topmargin      -1.5cm   % read Lamport p.163
\oddsidemargin  -0.04cm  % read Lamport p.163
\evensidemargin -0.04cm  % same as oddsidemargin but for left-hand pages
\textwidth      16.59cm
\textheight     23.94cm
\parskip         7.2pt   % sets spacing between paragraphs
\parindent         0pt   % sets leading space for paragraphs
\pagestyle{empty}        % Uncomment if don't want page numbers
\pagestyle{fancyplain}


\usepackage{Sweave}
\begin{document}
\Sconcordance{concordance:RoughDraft.tex:RoughDraft.Rnw:%
1 25 1 1 0 93 1}

\lhead{05-30-2019}
\chead{CSAFE - Hough Grooves Document Process}
\rhead{Page \thepage\ of \pageref{LastPage}}

\section{Introduction}

\section{Methods}

v bad

- Did not do full canny edge detection did not make a difference only increased processing time for groove identification

- Show plots with more thetas show difference between break off with only grooves


In order to best identify the GEAs we first want to diminish noise in the image. This can be achieved by converting each scan into an image gradient, which signifies where there are directional changes in the color of the image. This approach unforunately loses most of the detail of the three dimensional scans, however, it better highlights the differences between LEAs and GEAs. Once an image gradient is obtained we select only edges we consider to be "strong", meaning they have a magnitude above the 99th percentile. Our reason for not fully carrying out a Canny Edge detection algorithm before using a Hough transform is that the severity of the difference between GEAs and LEAs is so great that Canny Edge detection only makes minute improvements to the overall image for groove detection but increases processing time. Therefore it was deemed unnecessary for identifying Grooves in this case. Once we obtain the image gradient with only the strong edges we can then utilize our Hough transformation to obtain generally reasonable estimates of image boundaries. For the Hough transformation we utilize the function `` \texttt{hough\char`_lines}" from the imager package with the number of bins set to 100. DO I NEED TO TALK ABOUT WHAT HOUGH TRANSFORMS ARE. 

Once Hough lines are calculated we select only the lines that have scores greater than the $99.9^{th}$ percentile to try and ensure that we are dealing with only the strongest edges and do not accidentally pick up on any particularly strong striae on the LEAs. We also make the assumption that scans are oriented properly and as such, most Hough lines that correlate to the GEAs will be roughly vertical with some deviations in angle based on scanning technique. Therefore we select only the Hough lines that have angles from the positive x-axis less than $\frac{pi}{4}$. The parametrization outputted by `` \texttt{hough\char`_lines}" is of the form:

\begin{center}
$\rho = x \ cos(\theta) \ + \ y \ sin(\theta)$
\end{center}

Which requires a reparametrization in order to be used for groove identification. Thus we calculate the x and y intercept, the slope, and the average x-value of each Hough line. To determine which Hough line is the boundary of each GEA we rely on the assumption that the middle two thirds of each bullet scan is often part of the LEA. Therefore, we find the x-intercept of the lower and upper one sixth of our bullet scan then compare these boundaries to each Hough Line. The Hough lines whose average x values are the smallest distance from the x-intercept of the lower and upper one sixth of our bullet scan are defined to be the the start of our groove locations. 

\end{document}
