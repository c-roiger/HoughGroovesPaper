\documentclass[12pt]{article}
\usepackage{amsmath,amssymb,mathrsfs,fancyhdr,syntonly,lastpage,hyperref,enumitem,graphicx,subcaption, tikz, caption}
\usepackage[authoryear]{natbib} % Natbib options, [numbers] would give numerical citations.
\bibpunct{(}{)}{;}{a}{,}{,} % Tells it how you want references displaying in the text.

\usepackage[thmmarks,thref]{ntheorem}

\theoremstyle{nonumberplain}
\theoremheaderfont{\itshape}
\theorembodyfont{\upshape}
\theoremseparator{.}
\theoremsymbol{\ensuremath{\square}}
\newtheorem{solution}{Solution}

\hypersetup{colorlinks=true,urlcolor=black}

\topmargin      -1.5cm   % read Lamport p.163
\oddsidemargin  -0.04cm  % read Lamport p.163
\evensidemargin -0.04cm  % same as oddsidemargin but for left-hand pages
\textwidth      16.59cm
\textheight     23.94cm
\parskip         7.2pt   % sets spacing between paragraphs
\parindent         0pt   % sets leading space for paragraphs
\pagestyle{empty}        % Uncomment if don't want page numbers
\pagestyle{fancyplain}


\usepackage{Sweave}
\begin{document}
\Sconcordance{concordance:RoughDraft.tex:RoughDraft.Rnw:%
1 25 1 1 0 99 1}

\lhead{\today}
\chead{CSAFE - Hough Grooves Document Process}
\rhead{Page \thepage\ of \pageref{LastPage}}

\section{Introduction}

When firearms examiners try to match fired bullets they use a form of visual inspection to identify whether two bullets were fired from the same gun. As a bullet is being fired, manufacturing defects in the rifling of the gun should leave a unique set of striation. Firearms examiners utilize the striae to make identifications by matching striation patterns between bullets. A 2005 court case known as \textit{United States vs. Green}, established that firearms examiners cannot conclude that a specific bullet was fired from a specific gun ``to the exclusion of every other firearm in the world". That in fact the conclusion of a definitive match stretches beyond the data and methodology available to firearms examiners at the time.  Similarly, in 2009 the National Academy of Sciences published a report that criticized the scientific validity of forensic firearms identification and many other forensic methods. Both of these reports highlight the need for objective quantitative assessments of firearms identification. 

As three dimensional scanning has improved, new methods of bullet identification have developed that rely on established statistical or machine learning processes to automate bullet identification. One such example of automated bullet matching can be found in \cite{hare2017}, where the authors created an algorithm which removes unnecessary structure from bullet scans and made a random forest model that provides a probabilistic assessment of the strength of a match between two bullets. A key component of the algorithm which removes unnecessary bullet structure is the identification of "shoulder locations" of bullet lands to improve the extraction of bullet signatures. Below in figure \ref{fig:crosscut-motivation} is a two-dimensional visualization of a single crosscut from a bullet land. 



\begin{figure}[h!]
\begin{Schunk}
\begin{Sinput}
> x3p <- read_x3p("/Users/charlotteroiger/Documents/GitHub/HoughGroovesPaper/data/HS44 - Barrel 7 - Bullet 1 - Land 3 - Scan 1 - Sneox1 - 20x - auto light left image +20 perc. - threshold 2 - resolution 4 - Marco Yepez.x3p")